\documentclass[]{article}

\usepackage{include/homework}
%\renewcommand\title{The History of \LaTeX:\\ A Brief Introduction in a Form of a Homework}
%\renewcommand\author{\foreignlanguage{hungarian}{Hegyi Bálint Bence (Y9DYQD)}}

\title{A \LaTeX története:\\Rövid befezetés egy házi feladat formájában}
\author{\foreignlanguage{hungarian}{Hegyi Bálint Bence (Y9DYQD)}}
\hyphenation{Gutenbergig}

\begin{document}

\maketitle

\begin{center}
	\begin{minipage}{0.75\textwidth}
		\begin{abstract}
			Ez az írás a Dokumentumszerkesztés kurzusra (\emph{VIHIJV47}) készült a Budapesti Műszaki és Gazdaságtudományi Egyetemen, mint nagy házi feladat. A cikk célja, hogy leírja a \TeX, és \LaTeX történetét, a megalkotásának motivációjától, és megalkotásától a mai napjainkig. A cikk lényege hogy megmutassa az író képességét arra, hogy képes a tanultakat megfelelően alkalmazni az előbb említett nyelveken egy $\approx\!30.000$ karakteres munkával.
		\end{abstract}
	\end{minipage}
\end{center}

\vspace{2ex}

\begin{multicols}{2}
[
\section{A tipográfia alapjai}
\centering\parbox{0.75\textwidth}{\emph{Tipográfia: Nyomtatott betűkkel foglalkozó szakma és művészeti ág, melynek célja olyan írásképet kialakítani a betűtípusok és betűcsaládok alkalmazásával, mely egyszerre esztétikus és célszerű.}}
]
\justify

A tipográfia szó görög eredetű -- \foreignlanguage{greek}{τύπος} (tüposz): vert vagy vésett ábra, \foreignlanguage{greek}{γράφειν} (graphó): írni. A mai értelemben a 16. század közepétől használják, előtte a \emph{\foreignlanguage{italian}{scrivere sine penne}} (toll nélküli írás) kifejezés volt használatos. A nyugati tipográfia története Gutenbergig és a nyomtatásban használható mozgatható betűk feltalálásáig nyúlik vissza, de az írásképek kialakítása már évezredekkel ezelőtt külön művészeti és tudományos feladatot adott az embereknek.

A tipográfia a nyomtatott betűkkel foglalkozó szakma és művészeti ág, melynek célja olyan írásképet kialakítani a betűtípusok és betűcsaládok alkalmazásával, amely egyszerre esztétikus és célszerű. Modernebb megfogalmazásban a tipográfia az információ megjelenítésének szabályrendszere. De ezek a szabályok a hagyományon és a mindenkori ízlésen alapulnak, ezért folyamatosan változnak kisebb-nagyobb mértékben. Maga a szó is sok mindent jelentett az idők során: betűtervezést, betűmetszést, betűkkel való tervezést, nyomdászatot, nyomdai úton történő sokszorosítást. Ma a magyar nyelvben csak a betűkkel való tervezést, a szöveges közlés megformálását, szöveg és kép együttes elrendezését nevezzük tipográfiának.

A tipográfiával az a célunk, hogy a szöveg által hordozott információt minél teljesebben átadjuk az olvasó számára. A tipográfia szabályainak ismerete egyre lényegesebb egy olyan korban, amikor a szövegek ábrázolása és nyomtatása szinte bárkinek elérhető a számítógépek segítségével. A legtöbb ilyen „kiadvány”, melyeket szakképzettség nélkül szövegszerkesztőben vagy internetes weblapokon publikálnak, sajnos nem felel meg a tipográfia szabályainak, elveinek és fő céljának: az írás nyelvtanilag helyes, esztétikus kinézetű és könnyen olvasható legyen! A szabályok egy része megtalálható \emph{A magyar helyesírás szabályai} könyvben, más részük viszont csak nyomdászati szakkönyvekben.



\end{multicols}

\end{document}